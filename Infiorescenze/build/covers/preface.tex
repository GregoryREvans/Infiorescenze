\documentclass[11pt]{article}
\usepackage{fontspec}
\usepackage[utf8]{inputenc}
\usepackage{xunicode}
\setmainfont{Bell MT}
\usepackage[paperwidth=17in,paperheight=11in,margin=1in,headheight=0.0in,footskip=0.5in,includehead,includefoot,portrait]{geometry}
\usepackage[absolute]{textpos}
\TPGrid[0.5in, 0.25in]{23}{24}
\parindent=0pt
\parskip=12pt
\usepackage{nopageno}
\usepackage{graphicx}
\graphicspath{ {./images/} }
\usepackage{amsmath}
\usepackage{tikz}
\newcommand*\circled[1]{\tikz[baseline=(char.base)]{
            \node[shape=circle,draw,inner sep=1pt] (char) {#1};}}

\begin{document}


\begin{center}
\huge FOREWORD
\end{center}

\begingroup

\begin{center}
\leftskip3.25in
Infiorescenze is the Welsh equivalent to the English adjective \pmb{riparian}. A riparian area is an ecological region near water which is distinguishable from wetland. It is the region in between the land and a river. These regions serve to conserve soil and are biodiverse. ``Riparian'' comes from the Latin \textit{ripa} which means riverbank. While shaping the course of the river, the riparian zone is a liminal space, constantly changing over time and changing with the seasons or weather.
\rightskip\leftskip
\phantom{text} \hfill \phantom{()}
\end{center}

\endgroup

\vspace*{9\baselineskip}

\begin{center}
\huge PERFORMANCE NOTES
\end{center}

\leftskip1.25in
\pmb{Optional Scordatura} : This work may be performed in several tuning options, here listed in order of preference: \circled{1} the entire instrument tuned one octave down, \circled{2} string IV tuned one octave down, or \circled{3} the entire instrument remaining in standard tuning. While option 1 is preferred, such slackness can possibly result in the soundpost falling out of place if shaken. Option 2 represents a compromise if the performer deems the risk too great. Option 3 should be taken only in case other options are not feasible, for instance if the piece is programmed in a section of a concert where fast, stable retuning is not possible.
\rightskip\leftskip
\phantom{text} \hfill \phantom{()}

\leftskip1.25in
\pmb{Optional String Choices} : This work is intended to be performed entirely sul IV except where otherwise indicated. If the performer feels more comfortable crossing strings, an attempt should be made to keep this to a minimum even if neither scordatura option is employed.
\rightskip\leftskip
\phantom{text} \hfill \phantom{()}

\leftskip1.25in
\pmb{String Contact Points} : The indications of string contact positions such as $sul \ tasto$ (abbreviated as $T$), $sul \ ponticello$ (abbreviated as $P$), $molto \ sul \ tasto$ (abbreviated as $MST$), etc. should be considered as points along the continuum of the length string. The performer should make an effort to smoothly transition from one position to the next throughout the duration of the passage covered by the arrow-demarcated dashed line. When this arrow is not present, the performer should default to an $ordinario$ position.
\rightskip\leftskip
\phantom{text} \hfill \phantom{()}

\leftskip1.25in
\pmb{Bow Rotation Indications} : \circled{1} $col \ legno \ tratto$ is abbreviated as $clt.$ and \circled{2} $col \ legno \ batutto$ is abbreviated as $clb.$. When these abbreviations are not present, the performer should default to ordinary $crine$ bowing techniques.
\rightskip\leftskip
\phantom{text} \hfill \phantom{()}

\leftskip1.25in
\pmb{Spazzolato} : is notated with an arrow attached to the stems with the bowing direction indicated by the angle of the arrow. The pressure of this bowing is indicated by the dynamics. A soft brushing or swiping sound will be notated as a soft dynamic and a perforated or gritty sound will be notated as a loud dynamic. Spazzolato is distinct from circular bowing (which is notated with a circular arrow) in its overtone content and the perceptibility of notated pitch.
\rightskip\leftskip
\phantom{text} \hfill \phantom{()}

\leftskip1.25in
\pmb{Miscellaneous} : \circled{1} Tremoli should be performed as fast as possible and not as a measured subdivision of the duration to which they are attached. \circled{2} Diamond note heads represent a left hand finger pressure of a natural harmonic. \circled{3} Half-harmonic finger pressure is shown with a diamond half-filled with black for short durations and a diamond open on one end for long durations. \circled{4} An X note head is only used once in the piece to represent finger-percussion. This may be achieved with either hand, but preference should be given to whichever hand can produce the loudest taps (often the right hand). Pitch will be only slightly perceptible.
\rightskip\leftskip
\phantom{text} \hfill \phantom{()}

\leftskip1.25in
\pmb{Accidentals} : After temporary accidentals, cancellation marks are printed also in the following measure (for notes in the same octave) and, in the same measure, for notes in other octaves, but they are printed again if the same note appears later in the same measure, except if the note is immediately repeated.
\rightskip\leftskip
\phantom{text} \hfill \phantom{()}

\vspace*{15\baselineskip}

\begin{center}
\textit{Infiorescenze} was composed for Eric-Maria Couturier on the occasion of Mixtur Festival 2023.
\end{center}

\vspace*{10\baselineskip}

\begin{center}
duration: c. 2'30''
\end{center}

\vspace*{10\baselineskip}
.

\end{document}
